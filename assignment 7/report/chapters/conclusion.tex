
\section{Conclusion}
In this assignment, we implemented the Isolation Forest algorithm from scratch and evaluated its performance on a dataset provided by the instructor. The results showed that our scratch implementation achieved a PR curve area of 0.43 and an ROC-AUC of 0.673, which was slightly lower than the scikit-learn implementation.

We also explored dimensionality reduction techniques, such as t-SNE and PCA, to visualize the anomaly scores in a 2-dimensional space. However, these techniques did not significantly improve the anomaly detection performance of the Isolation Forest algorithm on this dataset.

The visualization of the Isolation Forest provided insights into how the algorithm partitions the data and isolates anomalies. The contour plots of the anomaly scores with t-SNE and PCA showed mixing of nominal and anomaly points, indicating the challenges in effectively separating them in the reduced-dimensional space.

Future work could involve exploring alternative anomaly detection techniques or investigating ways to preprocess the data to improve the performance of the Isolation Forest algorithm. Additionally, a more in-depth analysis of the dataset characteristics and the specific challenges it poses for anomaly detection could provide valuable insights for developing more robust and effective anomaly detection methods.
