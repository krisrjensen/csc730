\section{Introduction}
Anomaly detection is a crucial task in various domains, from fraud detection in financial transactions to identifying rare diseases in medical diagnosis. Traditional anomaly detection methods often focus on profiling normal instances and identifying instances that deviate from this normal profile. However, a novel approach called Isolation Forest (iForest) takes a fundamentally different approach by explicitly isolating anomalies instead of profiling normal points \cite{liu2008isolation}.

The requirements for this assignment are as follows\cite{assignment7}.\par
\begin{enumerate}
    \item Get the provided dataset from D2L.
    \item Write your own version of isolation forest code.
    \item Run your code on the dataset to obtain anomalousness scores for each point for your chosen parameter settings.
    \item Sort the points by anomalousness scores and generate a precision-recall curve.
    \item Generate the equivalent of the following figure from your forest.
    
\end{enumerate}

In this assignment, we implement the Isolation Forest algorithm from scratch and evaluate its performance on a dataset provided by the instructor. The dataset contains normal and anomalous instances, and our goal is to assess the effectiveness of the Isolation Forest algorithm in detecting anomalies.