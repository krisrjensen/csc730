The OptiGrid algorithms works on a dataset by first determining if the dataset can be contracted, dimensionally reduced, by projecting the data onto a d-1 space. 
This means that if the dataset can be divided into two seperate spaces by a plane, then the dataset can be seperated into at least two classes. Simply drawing a plane through the data is insufficient to determine the best plane to divide the data.\par

The algorithm will determine if a plane exists to divide the data by calculating the score of the plane. 
The score is determined by applying a kernel density estimation function to the data in the current dataset.
Depending on the kernel density esimation function and bandwidth, the score will contains some number of peaks.
If there are zero or one peaks, then the splitting for this dataset is complete. If the score is above a certain threshold, then the plane is considered a good plane to divide the data.
The data is then labeled as being left or right, or above or below the plane.\par
We will refer to left or top as A and right or bottom as B. 
The total dataset, D, is the union of A and B. Upon succesfull splitting of A and B, then the algorithm will recursively apply the same process to A and B.
The algorithm will continue to split the dataset until the dataset is no longer able to be split. 
