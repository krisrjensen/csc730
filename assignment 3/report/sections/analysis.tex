
Let's begin by setting the stage for the OptiGrid code, by analyzing the pseudocode set out by Hinneburg and Keim [2]. Then the structure of the code will be outlined, followed by a detailed analysis of each function and its purpose.\newline
% Compare this snippet from assignment%203/report/sections/discussion.tex:
\begin{tcolorbox}[breakable, title={$OptiGrid(dataset~D,~q,~min\_cut\_score)$}]
    %use numbers list
    \footnotesize     
    
    \begin{enumerate}[label=\arabic*., leftmargin=0.2cm]
        \item Determine a set of contracting projections P = \{$P_0$, . . ., $P_k$\}
        \item Calculate all projections of the dataset $D \to P_{0}(D)$, . . ., $P_{k}(D)$
        \item Initialize a list of cutting planes $BEST\_CUTS \leftarrow \emptyset, CUT \leftarrow \emptyset$
        \item FOR i=0 TO k Do
        \begin{enumerate}[label=\alph*., leftmargin=0.2cm]
            \item CUT $\leftarrow$ Determine best\_local\_cuts($P_i(D)$)
            \item CUT\_SCORE $\leftarrow$ score\_best\_local\_cuts($P_i(D)$)
            \item Insert all cutting planes with a score $\geq$ min\_cut\_score into $BEST\_CUTS$            
        \end{enumerate}
        END FOR
        \item IF $BEST\_CUT = \emptyset$ THEN RETURN $D$ as a cluster
        \item Determine the $q$ cutting planes with highest score from $BEST\_CUTS$ and delete the rest
        \item Construct a Multidimensional Grid $G$ defined by the cutting planes in $BEST\_CUTS$ and insert all data points $x \in D$ into $G$
        \item Determine clusters, i.e. determine the highly populated grid cells in G and add them to the set of cluster C
        \item REFINE(C)
        \item FOREACH Cluster $C_i \in C$ DO\newline
        OptiGrid($C_i$, q, min\_cut\_score)
    \end{enumerate}    
\end{tcolorbox}
\normalsize

\subsection{Functions of class OptiGrid}
\begin{enumerate}    
    \item $\_\_init\_\_$\newline
    {The $\_\_init\_\_$ functions acts as the class constructor. It initializes the class variables and sets the default values for the parameters. 
    The parameters include dataset dimension (\textbf{d}), number of cuts per iteration (\textbf{q}), the max cut score density of a plane ({$\mathbf{max\_cut\_score}$}), 
    noise level for dataset($\mathbf{noise\_level}$), several parameters related to the kernel density estimation including bandwidth ($\mathbf{kde\_bandwidth}$), 
    grid ticks ($\mathbf{kde\_grid\_ticks}$), sample size ($\mathbf{kde\_num\_samples}$), tolerance ($\mathbf{kde\_atol}$) and ($\mathbf{kde\_rtol}$), 
    and finally an argument for turning on or off output (\textbf{verbose}) . This function sets the initial conditions of the OptiGrid algorithm.}
    \item fit\newline
    {The fit function is the function that is called to start the OptiGrid algorithm. 
    It is the main function that calls all the other functions in the class.
    The fit function takes in the dataset and the initial weights as a parameter.\par
    The fit function first records the data length and initializes the list of clusters. Following this setup, the $\_iteration$ method is called which begins the OptiGrid algorithm.}
    
    \item $\_iteration$\newline
    {The first step in the $\_iteration$ function is to create an empty list of cuts. 
    The list of cuts is generated by looping through all dimensions of the dataset and calling the $\_find\_best\_cuts$ function. This loop will generate all cutting planes from d=1 to d=max(self.d). 
    The $current\_dimension$ paramter sent to the $\_find\_best\_cuts$ function is incremented by 1 each iteration.\par
    If the list of cuts is empty, then the function returns the dataset as a cluster and indicate there are no further cutting planes available for this dataset. 
    If the list of cuts is not empty, the list of cutting planes is sorted by score. 
    The cutting planes discovered in the previous step are then passed to GridLevel to construct a multidimensional grid. 
    The first call to GridLevel is made with the cutting planes list to construct the grid for this iteration in the recursive call stack. 
    Then a grid of cutting planes is created from the data and clusters. This grid data contains the information if the data is left or right of the cutting plane and encoded as either 0 or ${2^i}$\par
    At this point, the algorithm has created a grid and the data is labeled as being left or right, or above or below the plane. 
    This data needs to be iterated through to recursively apply the same process to the left and right datasets if the size of cluster exceeds 0.
    When the first call to $self.\_iteration$ is made, the algorithm will continue to split the dataset until the dataset is no longer able to be split. 
    Finally, when this first call returns, the algorithm will have found all the clusters in the dataset.}       


    \item $\_fill\_grid$\newline
    {}
    \item $\_create\_cuts\_kde$\newline
    {}
    \item $\_find\_best\_cuts$\newline
    {}
    \item $\_find\_peaks\_distribution$\newline
    {}
    \item $\_estimate\_distribution$\newline
    {}
    \item $score\_samples$\newline
    {}
    \item $\_score\_sample $\newline
    {}
\end{enumerate}

\subsection{Functions of class GridLevel}
\begin{enumerate}    
    \item $\_\_init\_\_$\newline
    {}
    \item $add\_subgrid$\newline
    {}
    \item $get_sublevel$\newline
    {}
\end{enumerate}


