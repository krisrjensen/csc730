The algorithm achieved mixed results. The clusters used for this example were spread out to ensure the workings of the algorithm were correct, but in a real world application, this would not always be the case.
The demo code used for the Optigrid implementation could be further improved if it allowed for cutting planes that are not all orthonormal to one another. In this way, more amorphous clusters could be split up across complex dimensions.

This project improved our knowledge of the Optigrid algorithm and of each of the algorithm's parts. Through documenting the example code given via deep description, we extended our understanding of the algorithm past that of just writing code about it.
By experimenting with the demo code, we also expanded our use case of external repositories and libraries in order to produce a new result. 
Overall, this project helped us learn more about the Optigrid algorithm and how to use it for unsupervised learning.