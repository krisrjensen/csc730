One requirment of our assignment was to review the clustering accuracy of the fuzzy c-means algorithm as applied to the skewed MNIST dataset.
The fuzzy c-means algorithm is a clustering algorithm that is based on the minimization of the objective function. 
The objective function is a function that is used to measure the quality of a clustering. The objective function was defined above in equation~\ref{eq:objective_function}.
When clustering is finished we need to compare the clusters that were produced to the actual labels of the data. We will use the rand index to compare the clusters to the labels. The rand index is defined as follows:
\begin{equation}
\label{eq:rand_index}
RI = \frac{n_{00} + n_{11}}{n_{00} + n_{11} +n_{01} + n_{10}} 
\end{equation}
where $n_{11}$ is the number of pairs of elements that are in the same cluster and in the same class, $n_{00}$ is the number of pairs of elements that are in different clusters and in different classes, and $C_{n}^{2}$ is the total number of pairs of elements in the dataset. The rand index is a value between 0 and 1. A value of 1 indicates that the clusters are identical to the labels. A value of 0 indicates that the clusters are completely different from the labels. The rand index is a good measure of the accuracy of the clustering algorithm. The rand index is defined in the scikit-learn library as the adjusted rand index. The adjusted rand index is defined as follows:
\begin{equation}
\label{eq:adjusted_rand_index}
ARI = \frac{RI - Expected(RI)}{max(RI) - Expected(RI)}
\end{equation}




