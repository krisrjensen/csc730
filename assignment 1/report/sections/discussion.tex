Rather than reducing the dimensions using previous statistical methods like PCA, our group’s hypothesis
was that most of the information contained in the image was found in the surface area of given pixel
values. The skewed MNIST dataset contained images with only white and black pixel values, thus the
surface area was captured using a specified median criterion of a typical [0,255] domain.
Our second hypothesis was that each class followed a distribution, and thus a difference from the mean
could be used as an anomaly score, thus the mean was subtracted from each count of white pixels
contained in each image. Lastly, this value was squared to ensure values further away from the mean
would give a larger score regardless of being greater or less than the mean.
As discussed in the results section, the accuracy for this method was 32.7~\%. To develop a baseline
comparison, a simulation was created which assigned random anomalous scores for each sample
achieved an anomalous accuracy of 33.3~\%, which shows our hypothesis for this method was incorrect. 

We then examined another method that utilized the FFT as the primary data analysis tool. This method utlimately calculatesd a score from the peaks that result in each power sprectral density.
Our results for this method exceeded the random baseline as well as the other methods we examined.