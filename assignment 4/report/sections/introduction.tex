The fourth assignment of this course is meant to introduce semi-supervised methods. Semi-supervised learning is a type of machine learning that uses a small amount of labeled data and a large amount of unlabeled data to train models. 
This is in contrast to supervised learning, which only uses labeled data, and unsupervised learning, which only uses unlabeled data. Semi-supervised learning is particularly useful when labeled data is scarce, as is often the case in practice [1].\par
The assignment is divided into three parts. The first part of the assignment is to select a supervised learning method then train the model on the full data set. 
The second part is to train the model on a small subset of the data, and the third part is to train the model on a small subset of data then use a wrapper method to implement semi-supervised learning.



