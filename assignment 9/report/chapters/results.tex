
\section{Results and Discussion}
\subsection{Rare Class Discovery Performance}
We ran the active learning rare class discovery algorithm on the provided datasets, tracking the number of classes discovered as a function of the number of queries. Figure~\ref{fig:classes_discovered} shows the results for both the active learning algorithm and a random query strategy.

%\begin{figure}[htbp]
%\centering
%\includegraphics[width=0.5\textwidth]{classes_discovered.png}
%\caption{Number of Classes Discovered vs. Number of Queries}
%\label{fig:classes_discovered}
%\end{figure}

The results indicate that the active learning algorithm was able to discover all the classes in the dataset with significantly fewer queries compared to the random query strategy. This demonstrates the effectiveness of the active learning approach in efficiently exploring the data and identifying rare classes.

\subsection{Insights from Data Visualization}
The 2D and 3D t-SNE visualizations provided valuable insights into the structure of the datasets. The 3D t-SNE plot, in particular, allowed for a more nuanced understanding of the data, revealing potential clusters and outliers that were not as apparent in the 2D projection.

The interactive 3D t-SNE movie further enhanced our ability to explore the data and gain a deeper understanding of the relationships between the different classes.